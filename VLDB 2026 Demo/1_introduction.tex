%!TEX root=paper.tex

\section{Introduction} 

\subsection*{Related Work}
\wc{@Elisa could you check the related work section if it looks good?}
% provenance graphs or relevant works
Provenance/lineage graphs~\cite{DBLP:journals/tods/ChapmanLMT24,
DBLP:journals/tvcg/CaiGXYWXWLW25,DBLP:conf/deem/PinaCK0M24} show the data flow
and transformations in various domains such as
databases~\cite{DBLP:journals/corr/abs-0912-5340, DBLP:conf/icdt/BunemanKT01},
and machine learning pipelines~\cite{DBLP:conf/deem/PinaCK0M24,
DBLP:journals/tods/ChapmanLMT24}. However, while they often focus on changes in
the data and transformations, they do not discuss what these transformations
mean in terms of metrics, and how the transformation meanings are related to the
final results. 
% visualizing data transformation
Visualization of intermediate
transformations~\cite{DBLP:journals/pvldb/LeeTABCKMSYHP21,
DBLP:conf/chi/WangLDM025} allow users to transform and explore data
interactively. DataFormulator2~\cite{DBLP:conf/chi/WangLDM025} provides an
environment for transforming and visualizing intermediate results, while
Lux~\cite{DBLP:journals/pvldb/LeeTABCKMSYHP21} surfaces intermediate insights
directly in Jupyter notebooks. However, neither discusses how transformations
relate to the final results in terms of metrics.